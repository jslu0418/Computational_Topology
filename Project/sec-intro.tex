%%% Local Variables:
%%% mode: latex
%%% TeX-master: "report"
%%% End:

At every moment, we are making plans of our actions for the very next instant.
Planning is involved in everywhere in our daily life, for example, which route we use to get to work, how to make a drone able to get back safely by itself.
Motion planning, is a problem that given a start position \(s\) of a object \(X\) and a goal position \(t\), possibly with a set of obstacles, computes a path that moves \(X\) from \(s\) to \(t\) without colliding with any obstacle.
It is also know as the piano mover's problem.
Imagine that we are given a computer-aided design (CAD) model of a house and a piano, and the goal is determine a way to move the piano from one room to another in the house without hitting anything~\cite{lavalle2006planning}.
This problem has many applications such as robotics, computational geometry, computer games, bioinformatics, virtual reality, and virtual prototyping~\cite{latombe2012robot}. Especially in the progress of autonomous robotics, it makes a critial role in the problem of enable the robotics to make decision for their own actions based on different cases~\cite{eric98}. These applications usually involve planning in planning under uncertainty~\cite{kaelbling2013integrated}, multi-agent systems~\cite{bourgault2002information}, and dynamically changing environment~\cite{van2005creating}.

While there are multiple versions of this problem according to how the obstacle information is given, researchers usually focus on the most simple version that all description of obstacles are given and fixed through the planning.
The problem in this setting is often referred as the basic motion planning problem, and it is usually solved by first building a graph to model the geometric structure of the environment and then finding a connected component that contains both the start and target positions. There are three common approaches to solve this basic setting problem: the roadmap approach, the cell decomposition approach, and the potential field approach~\cite{eric98}. The first two of them are both using the concept of \textit{configuration space}, \textit{free space} and the goal is finding a \textit{free path} where the configuration space is a transformation from the realistic space of the robot of certain shape and size into a space that the robot is shrinked to a point.

We first put topology away for a second and go through some motion planning algorithms. Schwartz and Sharir showed how the cylindrical cell decomposition can solve any motion planning problem with freedom degree \(k\), maximum degree of the \(n\) polynomial representation of the contact surface \(d\) in expected time \(O(nd^{3^k})\)~\cite{schwartz1983piano}. Using the roadmap technique, Canny gave an algorithm for any motion planning problem in \(O(n^k\log{n}d^{O(k^4)})\) deterministic time and \(O(n^k\log{n}d^{O(k^2)})\) expected time~\cite{canny1988complexity}.
While the object is convex polygon and the degree of freedom is low, there exists some more efficient algorithms. For example, motion planning for a convex object \(B\) translating amidst \(m\) convex and pairwise disjoint obstacles can be performed in \(O(m \log{m})\) time~\cite{sharir1997algorithmic}.

Inspired by the theory of topological complexity of algorithms for solving polynomial equations developed by Smale~\cite{smale2000topology} and Vassiliev~\cite{vassiliev1988cohomology}, Farber introduced a system of topological approaches for robot motion planning algorithms in 2003. They proposed the topological complexity \(\mathbf{TC}(X)\) of a topological space \(X\).
As a homotopy invariant, \(\mathbf{TC}(X)\) measures of the difficulty to plan a continuous motion of a robot in \(X\) if one considers the topological space as the configuration space of the mechanical system in a motion planning problem.
Their theory also introduced another notion of the topological complexity that analyzes the dependency of the instabilities of motion planning algorithms on homotopy properties of \(X\).
In the following years, applications of topological complexity and related theories to problemsin robotics developed to an independent area of research.
Farber \emph{et al.} study the problem of moving a line in \(\R^{n+1}\) by reduceing it to a topological problem of calculating the topological complexity of the real projective space \(\mathbf{TC}(\R\mathrm{P}^n)\) (recall the points of \(\R\mathrm{P}^n\) represent lines through the origin of \(\R^{n+1}\)) as well the immerial problem in \(\R\mathrm{P}^n\)~\cite{farber2002topological}.
In 2004, Farber and Yuzvinsky explored the topological complexity on designing collision free motion planning algorithms in Euclidean spaces. In the next year, Farber extended the result to the case of graphs~\cite{farber2004collision}.
and on graphs.
Moreover, there are also some research on topological complexity but less relative to our topic, the topology in motion planning.
Alexander Dranishnikov extend the theory to exploit the topological complexity of map in 2014~\cite{dranishnikov2015topological}. Later on, Pave\u{s}i\'c applied it to measure the manipulation complexity of robotic devices in kinematic maps ~\cite{pavevsic2017complexity,pavesic2018topologist} and showed how the topological complexities of covering projections approximate the topological complexity of the base space~\cite{pavevsic2018topological}.

This first topic that we discuss (Section~\ref{sec:complexity}) is the topological complexity of the topological space, which is also the main topic in this paper. Before we talk about the actually homotopy invariants of topological spaces, we begin with briefly introducing several settings of the motion planning problem and give the general definition of configuration spaces under those settings. After that, we introduce four different priori notions of topological complexity and what topology properties they can measure. More specifically, for one of those topological complexities, we further introduce some lower bounds of it for various spaces.

A topological map is a graph with its vertices represent distinct positions and edges represent a path between two distinct vertices. The concept is first proposed by Kuipers and Byung in 1991~\cite{kuipers1991robot} when they first tried to explore the topological approach to develop a robot exploration and mapping strategy by build a topological map to represent the environment. Kortenkamp and Weymouth (1994), Thrun and B\"ucken (1996) both studied practical strategies for producing topological maps~\cite{kortenkamp1994topological,thrun1996learning}. Ulrich and Nourbakhsh utilized the topological map to design a algorithm for topological localization~\cite{ulrich2000appearance}. On the other hand, it is usually hard to directly get a good topological map. Bosse \emph{et al.} introduced the idea to divide the free space into small regions and combine them together into a high level topological map~\cite{bosse2003atlas}.
In Section~\ref{sec:decomposition} we discuss the property of topological map and define good topological map that is convenient for motion planning, and strategies of partitioning the configuration spaces into simple regions for motion planning~\cite{DBLP:conf/isrr/ChosetR03}.

Denny \emph{et al.} proposed an approximate metric of measuring the homotopic similarity between two classes, and apply this metric to perform faster sample-based motion planning.