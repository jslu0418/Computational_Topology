% ---------
%  Compile with "pdflatex hw1".
% --------
%!TEX TS-program = pdflatex
%!TEX encoding = UTF-8 Unicode

% Template borrowed from Jeff Erickson.

\documentclass[11pt]{article}
\usepackage{jeffe,handout,graphicx}
\usepackage[utf8]{inputenc}		% Allow some non-ASCII Unicode in source
\usepackage{xcolor}
\usepackage{mdframed}
\usepackage{arydshln}
\usepackage{svg}

% =========================================================
%   Define common stuff for solution headers
% =========================================================
\Class{CS 7301.003}
\Semester{Fall 2020}
\Authors{2}
\AuthorOne{Dongpeng Liu}{dxl200000}
\AuthorTwo{Jiashuai Lu}{jxl173630}
%\Section{}
\DeclareMathOperator{\corners}{corners}
% =========================================================
\begin{document}
\HomeworkHeader{3}{1}% homework number, problem number
% ---------------------------------------------------------
\begin{enumerate}
    \item
      Let \(\Sigma\) be a combinatorial \(2\)-manifold, where each corner \(x\) of each face of
      \(\Sigma\) is assigned a positive real number \(\angle x\), called the \EMPH{angle} at \(x\).
      Let \(\corners(v)\) or \(\corners(f)\) denote the set of corners incident to a vertex \(v\) or
      a face \(f\), respectively.
      We define the \EMPH{curvature} of each vertex and face as follows:%
      \footnote{The definitions use individual passes around a \emph{circle} as the unit of angular
      measurements.}
      \[ \kappa(v) := 1 - \sum_{x \in \corners(v)} \angle x \qquad\qquad \kappa(f) := 1 - \sum_{x
      \in \corners(f)} (1/2 - \angle x).\]

      Prove the \EMPH{combinatorial Gau\ss-Bonnet theorem}:
      \[ \sum_{\text{vertex \(v\) of \(\Sigma\)}} \kappa(v) + \sum_{\text{face \(f\) of \(\Sigma\)}}
        \kappa(f) = \chi(\Sigma). \]

      \begin{solution}
        Let \(V(\Sigma),E(\Sigma),F(\Sigma)\) denote the vertex, edge, and face sets of a \(2\)-manifold \(\Sigma\).

        Observe that
        \begin{align*}
          \sum_{\text{vertex \(v\) of \(\Sigma\)}} \kappa(v) &=  \sum_{\text{vertex \(v\) of \(\Sigma\)}} (1-\sum_{x\in\corners(v)} \angle x)\\
          &=|V(\Sigma)|-\sum_{\text{corner \(x\) of \(\Sigma\)}}\angle x
        \end{align*}
        and since every edge contributes to two faces, we have
        \begin{align*}
          \sum_{\text{face \(f\) of \(\Sigma\)}} \kappa(f) &= \sum_{\text{face \(f\) of \(\Sigma\)}}(1-\sum_{x\in\corners(f)}(1/2-\angle x))\\
          &=|F(\Sigma)|-2|E(\Sigma)|/2+\sum_{\text{corner \(x\) of \(\Sigma\)}}\angle x
        \end{align*}
        together we get
        \begin{align*}
          \sum_{\text{vertex \(v\) of \(\Sigma\)}} \kappa(v) + \sum_{\text{face \(f\) of \(\Sigma\)}} \kappa(f)&=|V(\Sigma)|+|F(\Sigma)|-|E(\Sigma)|\\
          &=\chi(\Sigma)
        \end{align*}
      \end{solution}

  \item
    Suppose every face of \(\Sigma\) is a triangle.
    Prove the following special case of the combinatorial Gau\ss-Bonnet theorem:
    \[ \sum_{\text{vertex \(v\) of \(\Sigma\)}}(6 - |\corners(v)|) = 6\chi(\Sigma). \]
    \begin{solution}
      While every face of \(\Sigma\) is a triangle, we know \(3|F(\Sigma)|=2|E(\Sigma)|\). And it is easy to see that,
      \begin{align*}
        \sum_{\text{vertex \(v\) of \(\Sigma\)}}(6 - |\corners(v)|) &= 6|V(\Sigma)| - \sum_{\text{vertex \(v\) of \(\Sigma\)}}|\corners(v)|\\
                                                            &=6|V(\Sigma)|-2|E(\Sigma)|\\
                                                            &=6|V(\Sigma)|-3|F(\Sigma)|\\
                                                            &=6|V(\Sigma)|+6|F(\Sigma)|-9|F(\Sigma)|\\
                                                            &=6|V(\Sigma)|+6|F(\Sigma)|-6|E(\Sigma)|\\
                                                            &=6\chi(\Sigma)
      \end{align*}
    \end{solution}
  \item
    Now suppose \(\Sigma\) has boundary components.
    Let \(\chi(v)\) denote the number of edges incident to \(v\), counting loops twice, minus the
    number of corners incident to \(v\).
    We now redefine the curvature of a vertex \(v\) as follows:
    \[ \kappa(v) := 1 - \frac{\chi(v)}{2} - \sum_{x \in \corners(v)} \angle x. \]

    Prove that the combinatorial Gau\ss-Bonnet theorem holds in this more general setting.
    \begin{solution}
      Let \(|E(v)|\) denote the number of edges incident to \(v\), counting loops twice. Then we get
      \begin{align*}
        \sum_{\text{vertex \(v\) of \(\Sigma\)}} \kappa(v) &= |V(\Sigma)|-\sum_{\text{vertex \(v\) of \(\Sigma\)}}\frac{\chi(v)}{2}-\sum_{\text{corner \(x\) of \(\Sigma\)}}\angle x\\
                                              &=|V(\Sigma)|-\sum_{\text{vertex \(v\) of \(\Sigma\)}}\frac{|E(v)|-|\corners(v)|}{2}-        \sum_{\text{corner \(x\) of \(\Sigma\)}}\angle x
      \end{align*}
      combining this with the result in question 1.1, we have,
      \begin{align*}
        &\sum_{\text{vertex \(v\) of \(\Sigma\)}} \kappa(v) + \sum_{\text{face \(f\) of \(\Sigma\)}} \kappa(f)\\
        &=|V(\Sigma)|-\sum_{\text{vertex \(v\) of \(\Sigma\)}}\frac{|E(v)|-|\corners(v)|}{2}-\sum_{\text{corner \(x\) of \(\Sigma\)}}\angle x + |F(\Sigma)|\\
        &-\sum_{\text{face \(v\) of \(\Sigma\)}}\frac{|\corners(f)|}{2}+\sum_{\text{corner \(x\) of \(\Sigma\)}}\angle x\\
        &=|V(\Sigma)|+|F(\Sigma)|-\sum_{\text{vertex \(v\) of \(\Sigma\)}}\frac{|E(v)|}{2}\\
        &=|V(\Sigma)|+|F(\Sigma)|-|E(\Sigma)|=\chi(\Sigma)
      \end{align*}
      The second equation is from
      \(\sum_{\text{face \(v\) of \(\Sigma\)}}|\corners(f)|=\sum_{\text{vertex \(v\) of \(\Sigma\)}}|\corners(v)|\), and the fact that every edge contributes to the number of incident edges for two vertices implies the last equation.

\end{solution}
\item
    Suppose the surface \(\Sigma'\) is homeomorphic to a disk, and every face and \emph{interior}
    vertex of \(\Sigma'\) has curvature at most \(0\).
    Prove that at least three boundary vertices of \(\Sigma'\) have strictly positive curvature.
    \begin{solution}
      We know \(\chi(\Sigma')=1\), therefore we have
      \[\sum_{\text{vertex \(v\) of \(\Sigma'\)}} \kappa(v) + \sum_{\text{face \(f\) of \(\Sigma'\)}} \kappa(f)=1\]
      Let \(V_B\) be the set of boundary vertices of \(\Sigma'\).
      Then we get
      \[\sum_{v\in V_B} \kappa(v) + \sum_{v\in V(\Sigma')-V_B}\kappa(v)+\sum_{f\in F(\Sigma')} \kappa(f)=1\]
      We know \(\kappa(v)\le 0, \forall v\in V(\Sigma')-V_B\) and \(\kappa(f)\le 0, \forall f\in F(\Sigma')\). So \(\sum_{v\in V_B} \kappa(v)\ge 1\). From the definition of curvature of vertex \(v\) in question 1.3, we know if \(v\in V_B\),
      \begin{align*}
        \kappa(v)&=1-\frac{|E(v)|-|\corners(v)|}{2}-\sum_{x \in \corners(v)} \angle x\\
            &=1-\frac{1}{2}-\sum_{x \in \corners(v)} \angle x\\
            &=\frac{1}{2}-\sum_{x \in \corners(v)} \angle x
      \end{align*}
      where the second equation is from the number of edges incident to a boundary vertex is one more than the number of corners incident to the vertex.
      Since the angle of every corner is positive. Every boundary vertex has curvature strictly less than \(1/2\). Therefore, to make \(\sum_{v\in V_B} \kappa(v)\ge 1\), we need at least \(\floor{\frac{1}{1/2}}+1=3\) boundary vertices has positve curvature.
    \end{solution}
\end{enumerate}
\HomeworkHeader{3}{2}
  Let \(G\) be a cellularly embedded (i.e. every face is a disk) graph on a surface \(\Sigma\)
  \emph{with boundary}.
  Recall, a \EMPH{cut graph} is a subgraph \(H\) of \(G\) such that the closure of \(\Sigma
  \setminus H\) is a disk.
  A cut graph is minimal if no proper subgraph is a cut graph.
  For example, a minimal cut graph of an annulus is a path from one boundary to the other.

  A \EMPH{pair of pants} is a sphere minus three open disks.
  Let \(G\) be a graph with non-negatively weighted edges, cellulary embedded in a pair of pants
  \(\Sigma\).
  Describe an algorithm to find the minimum-length cut graph in \(G\) in \(O(n \log n)\) time.
  \emph{[Hint: What does a minimal cut graph of a pair of pants look like?]}
  \begin{solution}
    Let \(D\) denote the set of three open disks minus from the sphere.
    We know the cut graph will connect three boundary components. Therefore, it contains at least one boundary vertex for each boundary component. We basically want to find a minimum weight tree connecting three boundary components. So we can fix any disk \(d\in D\), then every time guess a vertex \(v\) on \(d\) and compute a minimum cut graph containing \(v\), and choose the minimum weight cut graph among those choices.
\begin{algo}
  \textul{\(\textsc{Minimum-Length Cut Graph }(G, D)\):}\+
  \\  Pick any disk \(d\in D\), let \(d_1,d_2\) be another two disks in \(D-\{d\}\)
  \\  Run MSSP for \(d\) as the source face.\+
  \\  During MSSP, every time when we pivot an edge, suppose current source is \(s_i\),\+
  \\  we update the shortest path \(p_{i1},p_{i2}\) from current source to \(d_1\) and \(d_2\).\-\-
  \\  Let \(s_k=\mathrm{Argmin}_{\text{\(s_i\) on \(d\)}}\text{sum weight of }p_{i1}\cup p_{i2}\)
  \\  Return the actual graph of \(p_{k1}\cup p_{k2}\)\-
\end{algo}
MSSP costs \(O(n\log{n})\), during the MSSP, when we update the shortest paths \(p_{i1},p_{i2}\), we only maintain the tail of the paths and the sum weight of the paths. So we did not increase the running time asymptotically.

After we know the pair of two endpoints of \(p_{k1}\) and \(p_{k2}\), we can use at most \(O(n\log{n})\) time to know the actual paths (we can afford to run MSSP again). Then we compute the union of them in \(O(n)\) time. So the total running time is \(O(n\log{n})\).

\end{solution}
\end{document}

%%% Local Variables:
%%% mode: latex
%%% TeX-master: t
%%% End:
