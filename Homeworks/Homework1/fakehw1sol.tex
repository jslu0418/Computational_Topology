% ---------
%  Compile with "pdflatex hw1".
% --------
%!TEX TS-program = pdflatex
%!TEX encoding = UTF-8 Unicode

% Template borrowed from Jeff Erickson.

\documentclass[11pt]{article}
\usepackage{jeffe,handout,graphicx}
\usepackage[utf8]{inputenc}		% Allow some non-ASCII Unicode in source
\usepackage{xcolor}
\usepackage{mdframed}
\usepackage{arydshln}
\usepackage{svg}

% =========================================================
%   Define common stuff for solution headers
% =========================================================
\Class{CS 7301.003}
\Semester{Fall 2020}
\Authors{2}
\AuthorOne{Dongpeng Liu}{dxl200000}
\AuthorTwo{Jiashuai Lu}{jxl173630}
%\Section{}

% =========================================================
\begin{document}
\HomeworkHeader{1}{1}% homework number, problem number
% ---------------------------------------------------------
\begin{enumerate}[(a)]\itemsep0pt
  \item
    Truthfully write the phrase \EMPH{\color{red} ``I have read and understand the course
    policies.''}

\begin{solution}
  I have read and understand the course policies.
\end{solution}

  \item[(b)]
    Prove that the composition of two PL homeomorphisms of the plane is another PL homeomorphism.

\begin{solution}
  Let \(\psi\) and \(\phi\) be any two PL homeomorphisms.
  Let \(\Delta_{\mathbb{H}}\) denote a triangulation of a PL homeomorphism \(\mathbb{H}\) such that the number of triangles in it is equal to the complexity of \(\mathbb{H}\).

  In order to show \(\theta=\psi\circ\phi\) is also a PL homeomorphism, we construct a triangulation \(\Delta\) of the plane such that the restriction of \(\theta\) to any triangle in \(\Delta\) is affine.

  First let \(\Delta=\Delta_{\psi}\).
  Since \(\Delta_{\psi}\) and \(\Delta_{\phi}\) are trigulations of the plane, every point of the plane is both in a unique triangle of \(\Delta_{\psi}\) and a unique triangle of \(\Delta_{\phi}\), and every triangle of \(\Delta\) overlaps with at least one triangle of \(\Delta_{\phi}\).

  For each triangle \(T\) of \(\Delta\), for every triangle \(T'\) of \(\Delta_{\phi}\) such that \(T\) and \(T'\) overlap, if the overlap of \(T\) and \(T'\) is not a line segment, it must be an \(n\)-gon where \(n\in\{3,4,5,6\}\), we add this \(n\)-gon to \(\Delta\) and create a frugal trigulation for this \(n\)-gon in \(\Delta\) if \(n>3\).

\begin{figure}[h]
  \centering
  \includesvg{fig/triangle}
  \caption{Possible n-gon by overlap.}
\end{figure}

  It is not hard to see that now \(\Delta\) is also a triangulation of the plane. Let \(f|_{R}\) denote the restriction of function \(f\) to a region \(R\). Let \(T\) be any triangle of \(\Delta\).
  Then \(T\) is either whole or part of the overlap of some triangle \(T'\) of \(\Delta_{\psi}\) and a triangle \(T''\) of \(\Delta_{\phi}\).
  Therefore, \(\Theta|_{T}\) to this triangle is the composition of two affine maps \(\psi|_{T'}\) and \(\phi|_{T''}\), i.e., \(\theta|_{T}=\psi_{T'}\circ \phi|_{T''}\).
  Since the composition of two affine maps are also affine, \(\Delta\) is a triangulation such that the restriction of \(\psi\circ \phi\) to any triangle of it is affine.
  The composition of two PL homeomorphisms of the plane is another PL homeomorphism.

\end{solution}

  \item[(c)]
    Suppose \(\phi\) is a PL homeomorphism with complexity \(x\) and \(\psi\) is a PL
    homeomorphism with complexity \(y\).
    What can you say about the complexity of the PL homeomorphism \(\psi \circ \phi\)?

\begin{solution}
  We reuse some notations in Solution(b).
  Given the procedure we used to build a triangulation for the PL homeomorphism \(\psi\circ \phi\) in Solution(b), let \(|\Delta|\) denote the number of triangles in triangulation \(\Delta\), we observe that the complexity of \(\psi\circ \phi\) will be \EMPH{at most}
  \[\min_{\Delta_{\psi},\Delta_{\phi}}|\Delta|\]
  where \(\Delta_{\psi}/\Delta_{\phi}\) is any triangulation of the plane such that the restriction of \(\psi/\phi\) to every triangle in it is affine, and \(\Delta\) is the triangulation we build using the procedure in Solution(b) with \(\Delta_{\psi}\) and \(\Delta_{\phi}\) as inputs.

  We could find an upper bound for \(|\Delta|\) given \(|\Delta_{\psi}|\) and \(|\Delta_{\phi}|\).
  For every triangle \(T\) in \(\Delta_{\psi}\), at most \(|\Delta_{\phi}|\) triangles of \(\Delta_{\phi}\) overlap with \(T\). Every overlap corresponds to at most \(4\) triangles in \(\Delta\) because the overlap part is at most a simple \(6\)-gon. Therefore, we have
  \[|\Delta|\le 4\cdot|\Delta_{\psi}|\cdot|\Delta_{\phi}|.\]
  and because there exist at least one triangulation \(\Delta_{\psi}\) with \(|\Delta_{\psi}|=x\) and one triangulation \(\Delta_{\phi}\) with \(|\Delta_{\phi}|=y\), we can see the \EMPH{complexity of} \(\psi\circ\phi\) is \EMPH{at most}
  \[\min_{\Delta_{\psi},\Delta_{\phi}}|\Delta|\le 4\cdot x\cdot y\]

  On the other hand, we think we cannot say anything about the lower bound of the complexity of \(\psi\circ\phi\). It could be only \(2\) while \(\psi=\phi^{-1}\).

  If we assume nothing about the relation between \(\psi\) and \(\phi\), the lower bound of the complexity of \(\psi\circ \phi\) is \(\ge\max\{x,y\}\), the equality holds when there exist \(\Delta_{\psi}\) and \(\Delta_{\phi}\) with \(|\Delta_{\psi}|=x,|\Delta_{\phi}|=y\) such that no edge in \(\Delta_{\psi}\) intersect edges in \(\Delta_{\phi}\) except at the endpoints.
\end{solution}

  \item[(d)]
    Prove that for any simple \(n\)-gon \(P\), there is a piecewise-linear homeomorphism \(\phi :
    \R^2 \to \R^2\) with complexity \(O(n)\) that maps the polygon \(P\) to a triangle.

\begin{solution}
\end{solution}

  \item[(e)]
    Prove that for any two simple \(n\)-gons \(P\) and \(Q\), there is a piecewise-linear
    homeomorphism \(\phi : \R^2 \to \R^2\) with complexity \(O(n^2)\) such that \(\phi(P) = Q\).

\begin{solution}
  From Problem(d), we know there exist two PL homeomorphism \(\phi_P,\phi_Q\) with complexity \(O(n)\) that map \(P\) and \(Q\) to the same triangle.
  Let \(\phi^{-1}_Q\) denote the inverse of \(\phi_Q\), the complexity of \(\phi^{-1}_Q\) should be same as \(\phi_Q\).

  Then we have \(\phi^{-1}_Q\circ\phi_P(P)=Q\).
  Let \(\psi=\phi^{-1}_Q\circ \phi_P\).
  From Solution(b) we know \(\psi\) is a PL homeomorphism that maps \(P\) to \(Q\).
  Let \(|\phi|\) denote the complexity of a PL homeomorphism \(\phi\).
  From Solution(c) we know the complexity of \(\psi\) is at most
  \[O(|\phi^{-1}_Q|)\cdot O(|\phi_P|)\le O(n^2)\]


\end{solution}
\end{enumerate}

% ---------------------------------------------------------
\HomeworkHeader{1}{2}% homework number, problem number
% ---------------------------------------------------------
\begin{enumerate}[(a)]\itemsep0pt
  \item
    Prove that every connected plane graph has either a vertex with degree at most \(3\) or a face
    with degree at most \(3\).

\begin{solution}
  We suppose there is a plane graph \(G\) of \(n\) vertices, \(m\) edges and \(f\) faces with every vertex has degree at least \(4\) and every face has degree at least \(4\).

  Every edge can be shared by at most two faces and at most two vertices in \(G\). So we know
  \[4n\le 2m,\:4f\le 2m\implies n+f-m\le 0\]
  According to Euler Formula, \(n+f-m=2\) as \(G\) is a connected plane graph. A contradiction. Therefore, every connected plane graph has either a vertex with degree at most \(3\) or a grace with degree at most \(3\).
\end{solution}

  \item[(b)]
    Prove that every simple bipartite planar graph has at most~\(2n - 4\) edges.

\begin{solution}
  We assume \(n\ge 3\) since if \(n=2\) the bipartite planar graph could have \(1\ge 2\cdot 2-4\) edge.

  Without loss of generality, we assume the graph is connected.
  If the graph has more than one connected components, for every connected component, suppose the number of vertices in it is \(n_i\), if the number of edges is at most \(2n_i-4\), then the number of edges in the graph is at most \(2n-4\).

  We prove the argument by considering the following two cases:
  \begin{enumerate}[1)]\itemsep0pt
  \item The graph only has one face.

    In this case, \(n+m-f=2\implies m=n-1\implies m\le 2n-4, \forall n\ge 3\)
  \item The graph has more than one face.
    In this case, every cycle in this graph is of even size, which means every face has degree \(\ge 4\) including the outer face.
    Every edge is shared by at most two faces.
    So \(4f\le 2e\implies e\ge 2f\).

    According to Euler Formula, \(2n+2f-2e=4\implies 2n-4=2e-2f\ge e\).
    Therefore, every simple bipartite planar graph has at most \(2n-4\) edges if \(n\ge 3\).
  \end{enumerate}
\end{solution}
\end{enumerate}

% ---------------------------------------------------------
\HomeworkHeader{1}{3}% homework number, problem number
% ---------------------------------------------------------
Let~\(G\) be an arbitrary plane graph, let~\(T\) be an arbitrary spanning tree of~\(G\), and
let~\(e\) be an arbitrary edge of~\(T\).
Color the vertices in one component of~\(T \setminus e\) red and the vertices in the other
component blue.
Prove that any face of~\(G\) is incident to either zero or two edges that have one red endpoint
and one blue endpoint.

\begin{solution}
  We say an edge \(\emph{bicolor}\) if it has one red endpoint and one blue endpoint.

  Let \(R,B\) denote two subtrees get from removing \(e\) from \(T\) such that \(R\) contains red vertices and \(B\) contains blue vertices.
  Let \(C\) be the edge cut between vertices of \(R\) and vertices of \(B\). All edges in \(C\) are bicolor and no edge in \(G\backslash C\) is bicolor.

  We know the dual of an edge cut of \(G\) is a dual cycle in \(G^*\).
  Let \(C^*\) be the dual cycle of \(C\) in \(G^*\).
  Every dual vertex in \(C^*\) is incident to exact two dual edges in \(C^*\), its corresponding primal face is incident to two bicolor edges. Every dual vertex in \(G^*\backslash C^*\) is incident no dual edge in \(C^*\), its corresponding primal face is incident to zero bicolor edge. The proposition is proved.


\end{solution}
\end{document}

%%% Local Variables:
%%% mode: latex
%%% TeX-master: t
%%% End:
