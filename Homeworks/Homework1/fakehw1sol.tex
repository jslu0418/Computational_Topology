% ---------
%  Compile with "pdflatex hw1".
% --------
%!TEX TS-program = pdflatex
%!TEX encoding = UTF-8 Unicode

% Template borrowed from Jeff Erickson.

\documentclass[11pt]{article}
\usepackage{jeffe,handout,graphicx}
\usepackage[utf8]{inputenc}		% Allow some non-ASCII Unicode in source
\usepackage{xcolor}
\usepackage{mdframed}
\usepackage{arydshln}

% =========================================================
%   Define common stuff for solution headers
% =========================================================
\Class{CS 7301.003}
\Semester{Fall 2020}
\Authors{2}
\AuthorOne{Author A. One}{aao123456}
\AuthorTwo{Jiashuai Lu}{jxl173630}
%\Section{}

% =========================================================
\begin{document}
\HomeworkHeader{1}{1}% homework number, problem number
% ---------------------------------------------------------
\begin{enumerate}[(a)]\itemsep0pt
  \item
    Truthfully write the phrase \EMPH{\color{red} ``I have read and understand the course
    policies.''}

\begin{solution}
  I have read and understand the course policies.
\end{solution}

  \item[(b)]
    Prove that the composition of two PL homeomorphisms of the plane is another PL homeomorphism.

\begin{solution}
  Kyle likes to start inline math mode using ``\textbackslash('' and end it using
  ``\textbackslash)'', but most online guides start and end math mode with ``\$''.
  Whatever you prefer is fine!
\end{solution}

  \item[(c)]
    Suppose \(\phi\) is a PL homeomorphism with complexity \(x\) and \(\psi\) is a PL
    homeomorphism with complexity \(y\).
    What can you say about the complexity of the PL homeomorphism \(\psi \circ \phi\)?

\begin{solution}
    Sometimes, you don't want to use inline math.

    \[1 < 2 = 6 / 3\]
\end{solution}

  \item[(d)]
    Prove that for any simple \(n\)-gon \(P\), there is a piecewise-linear homeomorphism \(\phi :
    \R^2 \to \R^2\) with complexity \(O(n)\) that maps the polygon \(P\) to a triangle.

\begin{solution}
  Sometimes, it's convenient to use pseudocode:
  \begin{algo}
    \textul{\(\textsc{SolveHomework}(problems[1..n])\):}\+
    \\  Set up boilerplate for writing homework.
    \\
    \\  for \(i \gets 1\) to \(n\)\+
    \\    Read problem \(problems[i]\)
    \\    for \(j \gets i\) down to \(1\)\+
    \\      \(\textsc{ThinkAboutProblem}(i)\)\-
    \\    Write solution to problem \(problems[i]\)\-
    \\
    \\  Turn in homework
  \end{algo}
\end{solution}

  \item[(e)]
    Prove that for any two simple \(n\)-gons \(P\) and \(Q\), there is a piecewise-linear
    homeomorphism \(\phi : \R^2 \to \R^2\) with complexity \(O(n^2)\) such that \(\phi(P) = Q\).

\begin{solution}
  Does this problem have too many parts?
\end{solution}
\end{enumerate}

% ---------------------------------------------------------
\HomeworkHeader{1}{2}% homework number, problem number
% ---------------------------------------------------------
\begin{enumerate}[(a)]\itemsep0pt
  \item
    Prove that every connected plane graph has either a vertex with degree at most \(3\) or a face
    with degree at most \(3\).

\begin{solution}
  I guess it's hard to fit lot of big objects in the plane.
\end{solution}

  \item[(b)]
    Prove that every simple bipartite planar graph has at most~\(2n - 4\) edges.

\begin{solution}
  Like really hard.
\end{solution}
\end{enumerate}

% ---------------------------------------------------------
\HomeworkHeader{1}{3}% homework number, problem number
% ---------------------------------------------------------
Let~\(G\) be an arbitrary plane graph, let~\(T\) be an arbitrary spanning tree of~\(G\), and
let~\(e\) be an arbitrary edge of~\(T\).
Color the vertices in one component of~\(T \setminus e\) red and the vertices in the other
component blue.
Prove that any face of~\(G\) is incident to either zero or two edges that have one red endpoint
and one blue endpoint.

\begin{solution}
  Sometimes a figure is useful.
  This is a topology course, after all.
  \begin{figure}[h]
    \centering
    \includegraphics[scale=0.3]{fig/temoc}
    \caption{Temoc is blue, like certain subsets of vertices.}
    \label{fig:temoc}
  \end{figure}
\end{solution}
\end{document}
